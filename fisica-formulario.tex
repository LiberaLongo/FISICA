\documentclass[70pt]{article}

\title{FORMULARIO FISICA}
\date{2023-02-13}
\author{Libera Longo}

%link utili:
% https://latex.codecogs.com/eqneditor/editor.php

\usepackage[a4paper,left=1cm,right=1cm,top=1cm,bottom=1cm]{geometry}	%margins
\usepackage{multicol}			%multicolonna
\setlength{\columnsep}{1cm}
\usepackage{xcolor}			%colore testo
\usepackage{amsmath}			%matematica formule
\usepackage{amsfonts}
\setcounter{secnumdepth}{0}		%no numerazione section e subsection
\usepackage{sectsty}			%section font

\sectionfont{\fontsize{10}{1}\selectfont}
\subsectionfont{\fontsize{8}{1}\selectfont}

\begin{document}
	%\maketitle
	\begin{small}
	\begin{multicols}{3}

\section{Cinematica}
	\subsection{Moto uniformemente accelerato}
	\subsection{Moto del proiettile}
	\subsection{Moto Circolare}
	\subsection{Moto Circolare Uniforme}
	\subsection{Moto Circolare Unif. accelerato}
	\subsection{Moto curvilineo}
\section{Sistemi a più corpi}
\section{Forze, Lavoro ed Energia}
	\subsection{Forze fondamentali}
	\subsection{Forze di Attrito}
	\subsection{Lavoro}
	\subsection{Energia}
\section{Imulso e Momento Angolare}
	\subsection{Equazioni cardinali}
\section{Leggi di conservazione}
\section{Urti}
\section{Moto Armonico}
\section{Momenti di inerzia notevoli}
\section{Gravitazione}
\section{Elasticità}
\section{Fluidi}
\section{Onde}
	\subsection{Onde su una corda}
	\subsection{Onde sonore}
	\subsection{Effetto Doppler}
\section{Termodinamica}
	\subsection{Primo principio}
	\subsection{Calore specifico}
	\subsection{Gas perfetti}
	\subsection{Macchine termiche}
	\subsection{Espansione termica dei solidi}
	\subsection{Conduzione e irraggiamento}
	\subsection{Gas reali}
\section{Calcolo vettoriale}
\section{Costanti fisiche}
	\subsection{Costanti fondamentali}
	\subsection{Altre costanti}
\section{Trigonometria}
\section{Derivate}
\section{Integrali}
\section{Approssimazioni ($x_0 = 0$)}

	\end{multicols}
	\end{small}
\end{document}
































