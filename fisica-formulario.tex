\documentclass{article}

\title{FORMULARIO FISICA}
\date{2023-02-07}
\author{Libera Longo}

%link utili:
% https://latex.codecogs.com/eqneditor/editor.php

\usepackage[a4paper,left=2cm,right=3cm,top=2.5cm,bottom=2.5cm]{geometry}	%margins
\usepackage{imakeidx}	%indice
\usepackage{xcolor}	%colore testo
\usepackage{amsmath}	%matematica formule
\usepackage{amsfonts}
\usepackage{parskip}	%paragraph with \\

\newcommand\norm[1]{\lVert#1\rVert}	%norma
\newcommand\separationline{\noindent\rule{\textwidth}{0.4pt}} %linea
%\renewcommand{\labelitemi}{$\bullet$}	%custom items
\renewcommand{\labelitemii}{$\circ$}	
\renewcommand{\labelitemiii}{$\diamond$}
%\makeindex

\begin{document}
	\maketitle
	%\printindex
	\section{MISURE}
	\section{MOTO IN UNA DIMENSIONE}
	\section{FORZA E LEGGI DI NEWTON}
	\section{MOTO IN DUE E TRE DIMENSIONI}
	\section{APPLICAZIONI DELLE LEGGI DI NEWTON}
	\section{QUANTITA' DI MOTO}
	\section{SISTEMI DI PARTICELLE}
	\section{CINEMATICA DEI MOTI ROTATORI}
	\section{DINAMICA DEI MOTI ROTATORI}
	\section{MOMENTO ANGOLARE}
	\section{ENERGIA 1: LAVORO ED ENERGIA CINETICA}
	\section{ENERGIA 2: ENERGIA POTENZIALE}
	\section{ENERGIA 3: CONSERVAZIONE DELL'ENERGIA}
	\section{LA GRAVITAZIONE}
	\section{STATICA DEI FLUIDI}
	\section{DINAMICA DEI FLUIDI}
	\section{FENOMENI OSCILLATORI}
	\section{FENOMENI ONDULATORI}
	\section{ONDE ACUSTICHE}
	\section{TEORIA DELLA RELATIVITA' RISTRETTA}
	\section{TEMPERATURA}
	\section{PROPRIETA' MOLECOLARI DEI GAS}
	\section{PRIMA LEGGE DELLA TERMODINAMICA}
	\section{ENTROPIA E SECONDA LEGGE DELLA TERMODINAMICA}
\end{document}


