\documentclass{article}

\title{fisica ESERCIZIARIO}
\author{Libera Longo}
\date{Febbraio 2023}

%link utili:
% https://latex.codecogs.com/eqneditor/editor.php

\usepackage[a4paper,left=2cm,right=3cm,top=2.5cm,bottom=2.5cm]{geometry}	%margins
\usepackage{amsmath}	%matematica formule
\usepackage{amsfonts}

\usepackage{multicol}			%multicolonna
\setlength{\columnsep}{1cm}
\usepackage[lastexercise]{exercise}     %eserciziario
\usepackage{float}
\usepackage{graphicx}						%figures

\graphicspath{{./images/}}

\begin{document}
    \maketitle
    %\begin{multicols}{3}
\section{Le misure}
\section{Moto in una dimensione}
    \subsection{I vettori e la cinematica}
    \subsection{Proprietà dei vettori}
        \begin{ExerciseList}
\Exercise[number=3]
    Il vettore $\vec{a}$ ha modulo 5,2 unità ed è orientato verso est. Il vettore $\vec b$ ha modulo 4.3 unità ed è orientato $35^{\circ}$ a est rispetto al nord. Costruendo i diagrammi dei vettori, trovate modulo, direzione e verso di (a) $\vec a + \vec b$ e (b) $\vec a - \vec b$.
\Answer
	\begin{figure}[H]
		\includegraphics{cap2_es3.png}
            \caption{esercizio 3 capitolo 2 $ | \vec a | = a = 5.2 $,  $ \theta_{\vec a} = 0^\circ $, $ | \vec b | = b = 3.3 $, $ \theta_{\vec b} = (90 - 35)^\circ = 55^\circ $}
            \label{fig:cap2_es3}
	\end{figure}
        
        $  $
        
                \end{ExerciseList}
    %\end{multicols}
\end{document}
